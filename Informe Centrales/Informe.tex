% ---------------PLANTILLA INFORMES-------------- %

%---------Preambulo-------
\documentclass[11pt,letterpaper]{extarticle}        % Clase


\usepackage[utf8]{inputenc}                      % Codificación UTF-8
\usepackage[spanish]{babel}                      % Idioma del documento
\usepackage[left=2cm,right=2cm,bottom=3cm,top=2.5cm]{geometry}  % Dimesiones


% --------------------------LIBRERIAS-------------------------- %

\usepackage{enumitem}               % Enumeracion
\usepackage{fancyhdr}               % Encabezados y pies de pagina
\usepackage[ampersand]{easylist}    % Listas
\usepackage{amsmath}                % Fórmulas matemáticas
\usepackage{amssymb}                % Símbolos matemáticos
\usepackage{caption}                % Leyendas
\usepackage{color}                  % Colores
\usepackage{fancyhdr}               % Encabezados y pié de páginas
\usepackage{float}                  % Administrador de posiciones de objetos
\usepackage{geometry}               % Dimensiones y geometría del documento
\usepackage{graphicx}               % Propiedades extra para los gráficos
\usepackage[hidelinks]{hyperref}    % Permite añadir enlaces y referencias
\usepackage[makeroom]{cancel}       % Cancelar términos en fórmulas
\usepackage[version=4]{mhchem}      % Fórmulas químicas
\usepackage{multicol}               % Múltiples columnas
\usepackage{lipsum}                 % Permite crear textos dummy
\usepackage{longtable}              % Permite utilizar tablas en varias hojas
\usepackage{listings}               % Permite añadir código fuente
\usepackage{setspace}               % Cambia el espacio entre líneas
%\usepackage{subfig}
\usepackage{subfig}              % Permite agrupar imágenes
\usepackage{titlesec}               % Cambia el estilo de los títulos
\usepackage{url}                    % Permite añadir enlaces
\usepackage{wrapfig}                % Permite comprimir imágenes

% LIBRERÍAS DEPENDIENTES
\usepackage{epstopdf}               % Convierte archivos .eps a pdf
\usepackage{multirow}               % Añade nuevas opciones a las tablas
\makeatletter

% INFORMACIÓN DEL DOCUMENTO
\newcommand{\nombredelinforme}{Pruebas en una Máquina de
Refrigeración Mecánica}
\newcommand{\fecharealizacion}{\today}
\newcommand{\fechaentrega}{\today}
\newcommand{\nombreuniversidad}{Universidad de Chile}
\newcommand{\nombrefacultad}{Facultad de Ciencias Físicas y Matemáticas}
\newcommand{\departamentouniversidad}{Departamento de Ingeniería Mecánica}
\newcommand{\imagendeldepartamento}{images/departamentos/dimec}
\newcommand{\imagendeldepartamentoescala}{0.29}
\newcommand{\localizacionuniversidad}{Santiago, Chile}
\newcommand{\nombredelcurso}{Laboratorio de Máquinas}
\newcommand{\codigodelcurso}{ME5301}



% CONFIGURACIONES
\newcommand{\tiporeferencias}{apa}                  % Tipo de referencias
\newcommand{\nombreltformulas}{Lista de Fórmulas}   % Nombre de la lista de fórmulas
\newcommand{\nombrelttablas}{Lista de Tablas}       % Nombre de la lista de tablas
\newcommand{\nombreltfiguras}{Lista de Figuras}     % Nombre de la lista de figuras
\newcommand{\nombreltcontend}{Índice de Contenidos} % Nombre del índice de contenidos
\newcommand{\nombreltwtablas}{Tabla}                % Nombre de las tablas
\newcommand{\nombreltwfigura}{Figura}               % Nombre de las figuras
\numberwithin{equation}{section}                    % Ecuaciones con numero de seccion

% -------------------------FUNCIONES ESPECIALES------------------------- %
\newcommand{\grados}{^{\circ}}                      % Circulo superior para grados
\newcommand{\quotes}[1]{``#1''}                     % Citas
\newcommand{\quotesit}[1]{\textit{\quotes{#1}}}     % Citas italico

% ------ FORMATO ------------%

\fancypagestyle{Portada}{
\fancyhead[L] {\nombreuniversidad \\ \nombrefacultad \\ \nombredelcurso \: \codigodelcurso }
\fancyhead[R]{\includegraphics[scale=\imagendeldepartamentoescala]{\imagendeldepartamento}}
\cfoot{}
}

\fancypagestyle{NoPortadaNoEnumerada}{
\fancyhead[L] {\nombredelinforme}
\fancyhead[R]{\nombreuniversidad}
\cfoot{}
}

\fancypagestyle{NoPortada}{
\fancyhead[L] {\nombredelinforme}
\fancyhead[R]{\nombreuniversidad}
\cfoot{\thepage}
}






% --------------------------------DOCUMENTO-------------------------------- %
% INICIO DEL DOCUMENTO
\begin{document}

%BEGIN_FOLD
% PORTADA
\newpage
\pagestyle{fancy}
\thispagestyle{Portada}
\vspace*{5cm}
\begin{center}
	\vspace{1cm}
	\noindent\rule{\linewidth}{0.4pt}\\
	\Huge {\textbf{\nombredelinforme}}
		\vspace{0.3cm} 
	\noindent\rule{\linewidth}{0.3pt}
\end{center}
\vfill

% INTEGRANTES, PROFESORES Y FECHAS
\begin{minipage}{0.965\textwidth}
	\begin{flushright}
		\begin{tabular}{ll}
			Alumno: 
				& \begin{tabular}[t]{@{}l@{}}
					Daniel Mardini González\\
				\end{tabular} \\
			Curso:
				& \nombredelcurso \\
			Código:
				& \codigodelcurso \\
			Profesor: 
				& \begin{tabular}[t]{@{}l@{}}
					Ricardo Díaz S.\\
				\end{tabular} \\
			
			\multicolumn{2}{l}{Ayudante del laboratorio: Pedro Pino T.} \\
			& \\
			\multicolumn{2}{l}{Fecha de entrega: \fechaentrega} \\
			\multicolumn{2}{l}{\localizacionuniversidad}
		\end{tabular}
	\end{flushright}
\end{minipage}

% CONFIGURACIÓN DE PÁGINA Y ENCABEZADOS
\newpage
\renewcommand{\listfigurename}{\nombreltfiguras}    % Nombre del índice de figuras
\renewcommand{\listtablename}{\nombrelttablas}      % Nombre del índice de tablas
\renewcommand{\contentsname}{\nombreltcontend}      % Nombre del índice
\renewcommand{\tablename}{\nombreltwtablas}         % Nombre de la leyenda de las tablas
\renewcommand{\figurename}{\nombreltwfigura}        % Nombre de la leyenda de las figuras
\pagestyle{fancy}
\thispagestyle{NoPortadaNoEnumerada}

% TABLA DE CONTENIDOS
\newpage
\tableofcontents        % Tabla de contenidos
\listoffigures          % Índice de figuras
\listoftables           % Índice de tablas


\newpage
\setcounter{page}{1}
\thispagestyle{NoPortada}

\section{Introducción}
\section{Objetivos}
\section{Antecedentes}
\section{Metodología}
\section{Datos obtenidos}
\section{Memoria de cálculo}
\section{Resultados}
\section{Análisis de resultados}
\section{Conclusiones}








\newpage
\section{Anexo}
\setcounter{section}{1}
\renewcommand*\thesection{\Alph{section}}

		% REFERENCIAS
\newpage
\begin{thebibliography}{99}
	\addcontentsline{toc}{section}{Referencias}
	
	\bibitem{pjd} 
		Pedro, Juan y Diego. 
		\textit{Como hacer informes en \LaTeX\.}. 
		Universidad de Chile, Facultad de Ciencias Físicas y Matemáticas, 2016.
	
	\bibitem{einstein}
		\hbadness=10000 El mismísimo Albert Einstein. 
		\textit{Otro titulo complicado e interesante}. 
		Análisis de físicas de informes, 322(10):891–921, 1905.
	
	\bibitem{ucursos} 
		También puedes agregar enlaces.
		\textit{U-Cursos al fin conoció \LaTeX.}
		2016.
		\\\url{https://www.u-cursos.cl/}
	
	\bibitem{conversortabla}
		Tables Generator.
		\textit{Convierte fácilmente tus tablas, o crea unas con un intuitivo editor de tablas.}
		\\\url{http://www.tablesgenerator.com/}
	
\end{thebibliography}
\end{document}

